\section{/\+Exemplos/clinfo/main.\+c}
Exemplo de clinfo utilizando a lib\+O\+P\+CL


\begin{DoxyCodeInclude}

\textcolor{preprocessor}{#include <stdlib.h>}
\textcolor{preprocessor}{#include <stdio.h>}
\textcolor{preprocessor}{#include <string.h>}
\textcolor{preprocessor}{#include <ctype.h>}
\textcolor{preprocessor}{#include <math.h>}
\textcolor{preprocessor}{#include <ctype.h>}
\textcolor{preprocessor}{#include "libopcl.h"}

\textcolor{preprocessor}{#ifdef \_\_APPLE\_\_}
\textcolor{preprocessor}{    #include <OpenCL/cl.h>}
\textcolor{preprocessor}{#else}
\textcolor{preprocessor}{    #include <CL/cl.h>}
\textcolor{preprocessor}{#endif}

\textcolor{keywordtype}{int} main(\textcolor{keywordtype}{int} argc, \textcolor{keywordtype}{char} *argv[])\{
    \textcolor{comment}{//Descobrir e inicializar as plataformas e Devices}
    \textcolor{comment}{//Segundo argumento -1 é usado quando se deseja descobrir todas plataformas disponíveis na máquina
       usada}
    lopcl_Init(lopcl_ALL, -1);
    \textcolor{comment}{//Imprime todas as informações disponíveis sobre a máquina usada}
    lopcl_PrintInfo(lopcl_ALL);
    \textcolor{comment}{//Libera os recursos usados durante a execução do programa}
    lopcl_Finalize();

\}
\end{DoxyCodeInclude}
 